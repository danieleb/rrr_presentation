% Created 2013-12-11 Wed 14:59
\documentclass[bigger]{beamer}
\usepackage[utf8]{inputenc}
\usepackage[T1]{fontenc}
\usepackage{fixltx2e}
\usepackage{graphicx}
\usepackage{longtable}
\usepackage{float}
\usepackage{wrapfig}
\usepackage{soul}
\usepackage{textcomp}
\usepackage{marvosym}
\usepackage{wasysym}
\usepackage{latexsym}
\usepackage{amssymb}
\usepackage{hyperref}
\tolerance=1000
\providecommand{\alert}[1]{\textbf{#1}}

\title{The Reproducible Research Repository}
\author{Daniele Barchiesi, Luis Figueira, Chris Cannam and Mark D Plumbley}
\date{2013-12-11 Wed}
\hypersetup{
  pdfkeywords={},
  pdfsubject={},
  pdfcreator={Emacs Org-mode version 7.9.3f}}

\begin{document}

\maketitle

\begin{frame}
\frametitle{Outline}
\setcounter{tocdepth}{3}
\tableofcontents
\end{frame}
\begin{frame}
\frametitle{The reproducible research repository}
\label{sec-1}


The Reproducible Research Repository (RRR) reduces the barriers for reproducible research by enabling sharing of documents and resources.

\begin{itemize}
\item Organised around \emph{experiments}, self-contained units corresponding to research results tipycally included in academic publications
\item Each experiment is linked to \emph{Datasets}, \emph{Software} and \emph{Publications}
\end{itemize}
\end{frame}

\end{document}
